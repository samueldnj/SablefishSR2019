% Document setup
\documentclass[11pt]{book}

% Location of the csas-style repository: adjust path as needed
\newcommand{\locRepo}{csas-style}

% Use the style file in the csas-style repository (sr.sty)
\usepackage{\locRepo/sr}

% header-includes from R markdown entry
\usepackage{pdflscape}

% Bibliography style file
% \bibliographystyle{../csas-style/res-doc}

%%%% Commands for title page etc %%%%%

% Title
\newcommand{\rdTitle}{Evaluating the robustness of candidate management procedures in the BC sablefish (\emph{Anoplopoma fibria}) for 2019-2020.}

% French title
\newcommand{\rdTitleFr}{}

% Title short
\newcommand{\rdTitleShort}{Robustness of sablefish MPs in BC}

% Publication year
\newcommand{\rdYear}{2019}

% Publication month
\newcommand{\rdMonth}{October}

% Report number
\newcommand{\rdNumber}{nnn}

% Approver (name\\position)
\newcommand{\rdApp}{Approver Name\\
Regional Director\\
Science Branch, Pacific Region\\
Fisheries and Oceans Canada}
% \newcommand{\rdYear}{20XX}
% \newcommand{\rdAppMonth}{January}
% \newcommand{\rdAppDay}{01}
\newcommand{\rdAppDate}{January 01, 20XX}

% Branch
\newcommand{\rdBranch}{Science Branch}

% Region
\newcommand{\rdRegion}{Pacific Region}

% Address
\newcommand{\rdAddress}{\textsuperscript{1}Pacific Biological Station\\
Fisheries and Oceans Canada, 3190 Hammond Bay Road\\
Nanaimo, British Columbia, V9T 6N7, Canada\\
\textsuperscript{2}Far, far away\\
Another Galaxy}

% Phone
\newcommand{\rdPhone}{(250) 756-7208}

% Email
\newcommand{\rdEmail}{\href{mailto:csap@dfo-mpo.gc.ca}{\nolinkurl{csap@dfo-mpo.gc.ca}}}

%%%% End of title page commands %%%%%
\begin{document}

\MakeFirstPage

\hypertarget{context}{%
\section{Context}\label{context}}

Since 2008, Fisheries and Oceans Canada (DFO) and the British Columbia (BC) groundfish fishing industry have collaborated on a management strategy evaluation (MSE) process intended to maintain a transparent and sustainable harvest strategy for sablefish fisheries in BC. Transparency and potential sustainability of candidate management procedures (MPs) are demonstrated by simulating MP performance against a set of pre-agreed conservation and socio-economic objectives. Operating models underlying the simulations are intended to represent key uncertainties related to sablefish stock status and productivity. The sablefish MSE process has been reviewed in several Canadian Science Advisory Secretariat processes, Canadian Science Advisory Secretariat Science Responses, and independent peer-reviewed scientific journals and books since 2008 (Cox and Kronlund \protect\hyperlink{ref-cox2008practical}{2008}; Cox et al. \protect\hyperlink{ref-cox2011management}{2011}, \protect\hyperlink{ref-cox2013roles}{2013}, \protect\hyperlink{ref-cox2019evaluating}{2019}; DFO \protect\hyperlink{ref-dfo2014performanc}{2014}). Canadian sablefish harvest advice derived from simulation-tested MPs has been adopted and subsequently approved by the Minister of Fisheries every year since 2011.

The sablefish MSE aims to follow a 3-year cycle in which the operating model is re-fitted to updated fishery and survey biomass indices, catch-at-age, at-sea releases, and tag release-recoveries. Each 3-year update also offers an opportunity to revise the conservation and fishery objectives, as well as to propose new candidate MPs.

Previous BC sablefish assessment and MSE work demonstrated that low recruitment, on average over the past three decades, has contributed to a long-term decline in spawning stock biomass and harvest opportunities. Stakeholder and management consultations identified at-sea release mortality of sub-legal sablefish (i.e., fish smaller than 55 cm size limit) as a potential source of mortality that, if reduced or avoided, would improve production of over-55 cm sablefish, spawning stock biomass, and, ultimately, future harvest opportunities (Cox et al. \protect\hyperlink{ref-cox2019evaluating}{2019}). While some voluntary management tactics aimed at reducing sub-legal mortality have been identified and implemented (e.g., improved fleet communication, informal move on rules, and increased electronic monitoring), to date management tactics aimed at reducing sub-legal mortality have not been formally evaluated through the sablefish MSE process. Indeed, closed-loop simulations suggest that both full avoidance and full retention of sub-legal sablefish may improve both average annual sablefish yield in directed fisheries as well as the probability of stock rebuilding to BMSY (Cox et al. \protect\hyperlink{ref-cox2011management}{2011}, \protect\hyperlink{ref-cox2019evaluating}{2019}). Unfortunately, full avoidance may not be feasible, especially in trawl fisheries, which encounter sub-legal sablefish as part of fishing operations for other species, while full retention may involve lost fishing opportunities (particularly for the trawl sector) and lower profitability for directed fisheries, because sub-legal sablefish are worth less per-kilogram than legal-sized fish. In consultations, industry stakeholders suggested that an ideal solution would involve incentives that shift fishing behaviour toward higher avoidance of sub-legal sablefish.

The DFO Fisheries Management Branch has, therefore, requested that the Science Branch (i) update the Sablefish operating model to include the most recent data available (up to 2018); (ii) update advice about expected performance of the current MP; and (iii) evaluate alternative MP and/or regulation options aimed at reducing productivity losses to sub-legal mortality. The key issue in (iii) is identifying MPs that minimize the impact of such regulations on fishing opportunities in non-directed fisheries (i.e., bottom trawl) where sub-legal sablefish are captured incidentally.

Advice arising from this Canadian Science Advisory Secretariat Science Response will be used to select a new MP for BC Sablefish for years 2020-2022 that is compliant with the DFO Sustainable Fisheries Framework and A fishery decision-making framework incorporating the Precautionary Approach policy (Fisheries and Oceans Canada \protect\hyperlink{ref-DFO2009}{2009}). In addition, this Science Response informs fishery managers and stakeholders about the fishery implications of limiting productivity losses due to sub-legal sablefish releases at-sea.

This Science Response Report results from the Science Response Process of September 2019 on evaluating the robustness of candidate management procedures in the BC sablefish (Anoplopoma fibria) fishery for 2019-2020.

\hypertarget{analysis-and-response}{%
\section{Analysis and response}\label{analysis-and-response}}

This Science Response uses a closed-loop simulation approach to evaluate the relative performance of candidate MPs for the BC sablefish fishery, using identical methodology to that presented in the previous MSE cycle (Cox et al. \protect\hyperlink{ref-cox2019evaluating}{2019}). The following sub-sections provide brief descriptions of the updated data provided for conditioning the sablefish operating model, the changes required to fit that data, and the new management procedure elements that were tested. Additional details of the simulation procedures, diagnostic checks, and performance measure calculations are given in Cox et al. (\protect\hyperlink{ref-cox2019evaluating}{2019}).

\hypertarget{objectives}{%
\subsubsection{Objectives}\label{objectives}}

The specific objectives of this Science Response are to:
\begin{enumerate}
\def\labelenumi{\arabic{enumi}.}

\item
  Describe operating model fits and inferences after fitting (conditioning) to updated biomass indices, catch-at-age, and new catch-at-age data derived from length-composition sampling of sablefish in the trawl fishery;
\item
  Derive a grid of 5 reference operating models and 5 robustness trial operating models based on uncertainties about Sablefish stock status and productivity (reference OMs) and year 2016 recruitment (robustness OMs);
\item
  Quantify and rank the relative performance of alternative MPs against updated Sablefish MSE Objectives (see Fishery Objectives section). Management procedure elements to be combined are given in Table 1.
\end{enumerate}
\hypertarget{methods}{%
\subsection{Methods}\label{methods}}

\hypertarget{updates-to-the-operating-model}{%
\subsubsection{Updates to the operating model}\label{updates-to-the-operating-model}}

Data updated to 2018 included biomass indices and catch-at-age for the stratifed random trap survey (StRS), catch-at-age for the commercial longline trap fishery, catch and total at-sea releases (in biomass units) for the commercial longline trap, longline hook, and trawl fisheries. New catch-at-age and catch-at-length datasets were obtained for the trawl fishery to help estimate trawl selectivity, which is the key determinant of sub-legal sablefish catch in trawl fisheries. The full trawl catch-at-age dataset (with some missing years) was derived from an age-length key given age and length data from 1972 to 2017.

A series of small changes were made to the operating model as part of routine attempts to improve fits to various data. These included (i) changing the functional form of trawl selectivity to a gamma density function, (ii) reducing the youngest model age class from age-3 to age-2 for all age composition series to better reflect the range of age-composition observations, (iii) adding new commercial trawl age-composition data (Appendix A), (iv) adding an (optional) estimated recruitment deviation in 2016, rather than using the expected recruitment off the stock-recruit curve, (v) updating the ageing-error matrix to use a simpler normal approximation a recommended in the previous CSAS review (Cox et al. \protect\hyperlink{ref-cox2019evaluating}{2019}); and (vi) imposing a standard deviation of \(\sigma = 0.1\) (on the log-scale) on trawl at-sea release observation errors to force a better fit to those data. Previous models avoided estimating recruitment in the 3 most recent years, mainly because (i) this would have been the first age-at-entry to the observations provided to the model and (ii) there is typically little information to support those estimates because fish are too small to be selected by the fisheries or surveys. However, for this update, we made change (iv) because we needed to improve fits to recent (very high) trawl at-sea release observations. Otherwise, we would be simulating effects of at-sea releases based on a model that could not adequately fit historical at-sea releases. This change has a potentially large impact on simulated MP performance and, therefore, is a focus of the robustness OMs (described below).

\hypertarget{operating-model-scenarios}{%
\subsubsection{Operating model scenarios}\label{operating-model-scenarios}}

\hypertarget{reference-oms}{%
\subsubsection{Reference OMs}\label{reference-oms}}

The reference OMs were derived using the same method as the previous MSE cycle (Cox et al.~2019). Briefly, we derive 5 OMs defined by the joint posterior distribution of 2018 spawning stock biomass (to reflect short-term biological risk) and stock-recruitment steepness (to reflect long-term stock productivity risk). The 5 combinations are chosen to represent the joint marginal mean of 2018 biomass and steepness and 4 outer points lying at the intersection of the mean of one variable, and the 10th and 90th percentiles of the marginal density of the other variable (Figure 1). For each of the 5 posterior points, the operating model was conditioned on a sample of 100 posterior draws constrained to lie within a Mahalanobis distance of 0.6 units from that point. We then used an empirical estimate of the posterior density at each of the 5 centres as a plausibility score for weighting MP performance across the 5 OMs within each of the reference and robustness sets (Table 2).

\hypertarget{robustness-oms}{%
\subsubsection{Robustness OMs}\label{robustness-oms}}

The robustness OMs were identical to the 5 reference OMs with the exception of how Year 2016 recruitment was treated in the OM historical conditioning and projections. The reference OM used draws from the joint posterior distribution (as defined above) for Year 2016 recruitment, which is approximately 12 million fish or about 5 times the historical average. For the robustness OMs, we simulated Year 2016 recruitment off the stock-recruitment relationship resulting in expected Year 2016 recruitment that was more similar to the long-term average (\(\sim 2.3\) million).

\hypertarget{fishery-objectives}{%
\subsubsection{Fishery Objectives}\label{fishery-objectives}}

Objectives for the B.C. Sablefish fishery have been developed iteratively via consultations between fishery managers, scientists, and industry stakeholders (Cox and Kronlund \protect\hyperlink{ref-cox2009evaluation}{2009}; Cox et al. \protect\hyperlink{ref-cox2011management}{2011}, \protect\hyperlink{ref-cox2019evaluating}{2019}; DFO \protect\hyperlink{ref-dfo2014performanc}{2014}). The five primary objectives guiding this fishery are:
\begin{enumerate}
\def\labelenumi{\arabic{enumi}.}

\item
  \textbf{P(fSSB \textgreater{} LRP)}: Maintain female spawning stock biomass (fSSB) above the limit reference point \(LRP = 0.4B_{MSY}\), where \(B_{MSY}\) is the operating model female spawning biomass at maximum sustainable yield (\(MSY\)), in 95\% of years measured over two sablefish generations (36 years);
\item
  \textbf{P(decline)}: When female spawning stock biomass is between \(0.4B_{MSY}\) and \(0.8B_{MSY}\), limit the probability of decline over the next 10 years from very low (5\%) at \(0.4B_{MSY}\) to moderate (50\%) at \(0.8B_{MSY}\). At intermediate stock status levels, define the tolerance for decline by linearly interpolating between these probabilities;
\item
  \textbf{P(fSSB \textgreater{} \(B_{MSY}\))}: Maintain the female spawning biomass above a target level of (a) \(B_{MSY}\) when inside the healthy zone, or (b) \(0.8B_{MSY}\) when rebuilding from the Cautious zone, in the year 2052 with a probability of 50\%;
\item
  \textbf{P(Catch \textgreater{} 1,992 t)}: Minimize probability that annual catch levels are below 1,992 tonnes measured over two sablefish generations.
\item
  \textbf{MaxCatch}: Maximize the average annual catch over 10 years subject to Objectives 1-4.
\end{enumerate}
Performance measures corresponding to Objectives 1-4 (in bold) are read as ``Probability of (condition)''. Performance measures are calculated for each simulation replicate, and the expected performance for a management procedure is summarized by the mean (or median) over the 100 replicates of each simulation. Full details of performance measures and calculations are given in Cox et al. (\protect\hyperlink{ref-cox2019evaluating}{2019}).

As noted above, there is a price premium for larger size classes of sablefish, which means that the same tonnage of landed catch may yield widely different dockside values if the underlying size distributions of individual fish are substantially different. This may have consequences for sub-legal regulation options that require landing small sablefish (e.g., full retention). Therefore, in addition to presenting catch performance statistics (e.g., Objective 5), we also computed cumulative landed value over 10 years and average landed value per tonne by fleet (because the size composition of the catch also differs by fleet).

\hypertarget{management-procedures}{%
\subsubsection{Management procedures}\label{management-procedures}}

A management procedure (MP) represents a specific, repeatable algorithm for computing annual total allowable catches (TACs) in a fishery. In most cases, MPs involve monitoring data, assessment methods for processing data and estimating stock status, harvest control rules for translating assessment outputs into catch limits, and meta rules that may include constraints on TAC changes, as well as conditions (e.g., exceptional circumstances) for triggering deviations from the standard MP harvest advice.

The MP currently used to set annual sablefish TACs was initially developed in Year 2011 and revised in two subsequent MSE iterations. Generally, the MP consists of (i) 3 biomass indices; (ii) a surplus production model with observation and process errors for estimating stock biomass from the biomass indices; (iii) a 60:40 harvest control rule (HCR) in which the maximum target harvest rate is adjusted from 5.5\% when estimated biomass is above 60\% of estimated \(B_{MSY}\), and 0\% when the estimated biomass is below 40\% of \(B_{MSY}\); (iv) a meta rule stating that TAC increases are 0 unless the HCR recommended increase is more than 200 tonnes (TAC decreases are always adopted); and (v) a meta rule adjusting the maximum target fishing mortality rate from 9.5\% in Year 2017 to 5.5\% in Year 2021. Complete details of the current MP specifications are given in Cox et al. (\protect\hyperlink{ref-cox2019evaluating}{2019}).

For this Science Response, we evaluated performance of the current MP for sablefish, a NoFishing reference case, and 13 variations of the current MP that only change at-sea release regulations. The MP variants are constructed by combining 3 features:
\begin{enumerate}
\def\labelenumi{\arabic{enumi}.}

\item
  at-sea sub-legal release cap in which all at-sea releases below the cap may be released without penalty and amounts exceeding the cap go to overages. Caps are noCap, 0\%, 50\%, 100\%, and 150\% over the average (VALUE) at-sea releases that occurred between 2006 and 2018. The current MP involves no cap (unlimited at-sea releases without penalty), while a full retention (frt) case allows no at-sea releases (all fish brought on-board vessels must be landed and counted against the TAC).
\item
  fixed allocations of the total at-sea release cap to each fleet (i.e., trap, longline hook, trawl). Allocations are computed based on either recent (rct = \((23\%, 18\%, 59\%)\) ) or historical (hst = \((30\%, 37\%, 33\%)\) ) fleet-specific average proportions of the total annual at-sea releases.
\item
  amortization period of either 5 (am5) or 10 (am10) years over which to spread at-sea release overages to future TACs.
\end{enumerate}
Our naming convention for MPs attempts to describe the at-sea release regulations by concatenating CAP\_ALLOCATION\_AMORTIZATION settings. For example, the \textbf{cap.5\_hstAl\_am5} MP involves a total at-sea release cap that is 50\% of the historical average (\textbf{cap.5}), a cap allocation among fleets that is computed from the historical, fleet-specific average proportions (\textbf{hstAl}), and a 5-year amortization period for at-sea release overages (\textbf{am5}). The two special cases are the current MP (\textbf{noCap}), which has no cap, and full retention (frt), which has no releases.

\hypertarget{a-worked-example-at-sea-release-regulation-for-cap.5_hstal_am5.}{%
\subsubsection{\texorpdfstring{A worked example at-sea release regulation for \textbf{cap.5\_hstAl\_am5}.}{A worked example at-sea release regulation for cap.5\_hstAl\_am5.}}\label{a-worked-example-at-sea-release-regulation-for-cap.5_hstal_am5.}}

Here we provide the sequence of calculations used to establish annual at-sea release caps and then how they affect future TAC allocations. In the computations below, the following notation applies: \(t\) is Year, \(g\) is fleet, \(p(g)\) is allocation proportion for fleet \(g\), and \(\bar{R}\) is the average total (over all fleets) at-sea releases between 2006 and 2018.
\begin{enumerate}
\def\labelenumi{\arabic{enumi}.}

\item
  Calculate 50\% at-sea release CAP for year and fleet: \begin{equation*}
        CAP(t,g) = 0.5 \cdot \bar{R} x p(g).
  \end{equation*}
\item
  Run simulation for year t to get actual at-sea releases: \(R(t,g)\).
\item
  Calculate overage \(o(t,g)\) for the year as the difference between actual releases \(R(t,g)\) and the \(CAP(t,g)\): \(o(t,g) = R(t,g) - CAP(t,g)\)
\item
  Amortization period is 5 years, so add 1/5th of this year's overage. to the accumulated overage account \(O(t,g)\) in each of the next 5 years: \begin{equation*}
  O(t',g) = O(t',g) + o(t,g)/5, \mbox{ for } t' = t+1, ..., t+5.
  \end{equation*}
\item
  Get adjusted legal-sized sablefish TAC for next year by subtracting overage account for that year from initial TAC' (TAC' set by the MP prior to at-sea regulations): \(TAC(t,g) = TAC'(t,g) - O(t,g)\).
\end{enumerate}
This approach aims to create an incentive to avoid sub-legal sablefish via future TAC reductions (using a one-for-one accounting of sub-legal biomass to legal sized sablefish biomass), while also allowing some flexibility year-to-year for unpredictably large at-sea releases in any given year. Note that the overage account can never be less than zero, so that TACs cannot be increased above the initial TAC set by the first stage MP.

\hypertarget{management-procedure-tuning}{%
\subsubsection{Management procedure tuning}\label{management-procedure-tuning}}

There are five primary dimensions of MP performance against objectives. The first three represent biomass conservation performance against the LRP, short-term probability of decline, and achieving a long-term target near \(B_{MSY}\), while the fourth and fifth dimensions relate to maintaining catch levels above an industry-preferred floor and short-term average catch. It is rare that two MPs would have comparable performance across four of these objectives while only differing on one. If this were the case, then MP decisions would be straightforward -- choose the MP with better performance on the fifth criterion. Unfortunately, MPs typically differ on all 5 dimensions simultaneously, which makes it difficult to compare performance without, at least, establishing some equivalency between conservation probabilities (performance dimensions 1-3) and short-term average catch (performance dimension 5).

Management procedure tuning provides a means of establishing equivalent MP performance against objectives for which the values and probabilities are well established. For example, maintaining the sablefish stock above the LRP (\(0.4B_{MSY}\)) with high probability has not been openly debated since it is an overarching Canadian policy directive in the sablefish fishery context (at least not debated over the 10+ year history of the sablefish MSE). Similarly, maintaining a low probability of short-term decline has also not been debated, probably because avoiding further decline has been the key overriding objective of the sablefish fishing industry since the inception of the MSE process. Objective 3 -- spawning biomass in the healthy zone within 2 generations -- has been debated over the years for practical reasons. Specifically, there is concern that achieving Objective 3 would require severe short-term catch restrictions for highly uncertain long-term benefits. Over the past year, the sablefish industry and DFO Science and Management agreed to revise Objective 3 to achieve biomass in the healthy zone by a specific end-year (2052) with at least 50\% probability, i.e., median fSSB at, or above, \(B_{MSY}\). As we demonstrate below, this objective is now feasible given sablefish dynamics and also achievable for a range of realistic MPs. However, this raises a new question: how much is it worth (i.e., in catch) to improve Objective 3 performance from, say, \(P(B_{2052} \geq B_MSY) = 0.5\) to \(P(B_{2052} \geq B_MSY) = 0.55\)? The probability difference of only 5 percentage points could mean a difference of several hundred tonnes in average annual catch, which would cumulatively added up to tens of millions of dollars in landed value. MPs that are better at Objective 3 do so at the expense of Objectives 4 and 5.

We aimed to simplify interpretation of MP performance by tuning all MPs to a standard \(P(B_{2052} \geq B_MSY) = 0.5\), which ensures that all MPs under consideration meet all the stated conservation objectives under the reference OMs and only differ in their catch performance. This simplification is needed in the current context, because the at-sea regulations we evaluated have catch and fishing opportunity implications across fishing sectors.

Tuning was achieved by iteratively adjusting Year 2022 phased-in maximum target fishing mortality rates \(F_{2022}\) until each MP met the lower limit of Objective 3, i.e., \(P(B_{2052} \geq B_{MSY}) = 0.5\). We tuned the surplus production model management procedures separately to the reference OM (\textbf{hiRec2016}) and robustness OM (\textbf{simRec2016}) scenarios, leading to different \(F_{2022}\) values under each recruitment scenario. Note that this target maximum harvest rate replaces the final maximum target rate of 5.5\% after the 5-year phase-in from 2017 (Cox et al. \protect\hyperlink{ref-cox2019evaluating}{2019})

The most critical, and perhaps unrealistic in some cases, assumption in the above is that fleets stop fishing when their fleet-specific TACs are fully landed.

\hypertarget{results}{%
\subsection{Results}\label{results}}

\hypertarget{operating-model-update-and-implications-for-stocks-status}{%
\subsubsection{Operating model update and implications for stocks status}\label{operating-model-update-and-implications-for-stocks-status}}

\hypertarget{management-procedure-evaluation-results}{%
\subsubsection{Management Procedure Evaluation Results}\label{management-procedure-evaluation-results}}

\hypertarget{reference-oeprating-model-set}{%
\subsubsection{Reference oeprating model set}\label{reference-oeprating-model-set}}

Year 2016 recruitment is a main driver of spawning biomass and fishery outcomes in the reference OM simulations. Given the high 2016 recruitment, spawning biomass increases rapidly over the first 5 years of the projection period as these fish become fully recruited to the fisheries and then the spawning biomass (Fig). Spawning biomass then trends downward toward \(B_{MSY}\) in the long-term as this large year class is fished down. Under these conditions, all MPs met all the conservation criteria defined by Objectives 1-3. The probability of catch below the 1,992 t floor was 2.6\% or less across all MPs. Tuning MPs to meet Objectives 1-3, and specifically treating Objective 3 as a target, highlights MP performance differences in average annual catch over the next 10 years.

As expected, management procedures with more restricted at-sea release regulations ranked higher in terms of 10-year average catch (Table X) with the values ranging from 4,510 t per year for full retention (\textbf{frt}) to 3,700 t per year for a cap 150\% higher than average, recent cap allocation among fleets (i.e., allocating 59\% to trawl), and 5-year amortization (MP13 \textbf{cap1.5\_rctAl\_am5}). This difference is attributable to two factors. First, the key assumption here is that fishing activity stops once the TAC is reached, so full retention involves less mortality of sub-legal fish over all fleets. This leads to a large reduction in growth overfishing for the full retention regulation -- gains in sablefish body growth are much higher than losses due to natural mortality in sub-legal size classes -- and, therefore, average weight of legal-sized fish in the catch is larger. Second, the fishery can operate at higher fishing mortality rates because survival over sub-legal size classes is higher and therefore higher survival to fisheries and the spawning stock. Indeed, the current MP maximum target \(F = 5.5\%\) is largely the result of lower survival through sub-legal size classes, which inhibits MPs from meeting the future spawning biomass Objective 3. In contrast, the full retention MP meets Objective 3 despite a maximum target 4. \(F = 7.4\%\) (Table X).

Differences in average annual catch were smaller among at-sea regulation options not involving full retention. An at-sea release cap of 50\% of the historical average resulted in average annual catch levels approximately 200 t higher than the current MP (\textbf{noCap}), regardless of how the cap in the former was allocated among fleets (MP3 and MP5 vs MP15; Table X).

An at-sea release cap equal to 100\% of the historical average also produced 200 t more average annual catch compared to the current MP, as long as the cap was allocated according to the historical at-sea release proportions and amortized over 5 years. The similarity to the lower 50\% caps described above mainly reflects cap allocation to the trawl fleet, where the recent allocation (57\%) is approximately twice the historical (29\%), so switching to the lower, historical allocation allows for doubling the cap, i.e., the total at-sea release amounts allocated to the trawl fleet are similar. In general, the historical allocation options ranked higher than the recent allocations because the historical allocation involves lower at-sea releases by the trawl fleet. The amortization period did not have as noticeable an effect as the overall cap and allocation options, in that order.

Increasing the cap to 150\% of the historical average produced the lowest average annual catch, despite the current MP having no cap at all. Although average 10-year catches were similar, at-sea releases in the current MP (\textbf{noCap}) change mainly with recruitment and therefore have less impact than a 150\% cap, which decouples at-sea releases and recruitment to some (small) degree and allows trawl fishing to continue past current sub-legal catch rates.

For higher caps and recent at-sea release allocation, the effect of amortization switched from 5 years being better to 10 years being better. Although the differences were small (MP 12 vs MP13; Table X), the switch probably occurs because there is little to no growth overfishing benefit of amortization at high caps and recent allocations, which would mean higher trawl releases than present. In this case, the amortization period has a direct effect on TACs with longer amortization periods having less impact because any overages are spread over the longer period.

We initially expected that a full retention and/or lower cap regulations would negatively affect fishery landed value because the landed catch would consist of higher proportions of sub-legal fish. Price premiums for sablefish (see table below; C. Acheson per comm., Spring 2019) may result in several dollars per pound differences between sub-legal (\textless{} 3 lbs) and legal-sized sablefish.

\textbf{TABLE OF SABLEFISH VALUES}

Indeed, the average landed value per tonne was approximately \$170 lower for a full retention trap fishery compared to any of the other at-sea release regulations (Table X), while landed value was approximately \$20 and \$90 per tonne lower for longline hook and trawl landings, respectively. Size-selectivity for trap, and especially longline hook, fisheries is shifted far enough toward larger sizes that the impacts of retaining smaller fish are relatively small compared to the benefits of higher average TACs. Indeed, cumulative landed values over ten years were \$47 million, \$18 million, and \$15 million higher for trap, longling hook, and trawl fisheries under full retention fishery compared to the next best average annual catch regulation option (i.e., MP3, \textbf{cap.5\_hstAl\_am5}).

The next best at-sea release regulation option after full retention was very different between trap and longline hook fisheries and trawl. For instance, as noted above MP3 (cap.5\_hstAl\_am5) was the next best option for trap and longline hook, in terms of both average annual TAC and cumulative value (Table X). In contrast, the next best option for trawl landed value was MP13 (\textbf{cap1.5\_rctAl\_am5}), which had the lowest average annual TAC. The value difference for trawl between this option and full retention was only \$5 million over 10 years, while the value differences between MP3 and MP13 for trap and longline hook were \$33 million and \$32 million, respectively. Thus, trap and longline hook fisheries benefit from more restrictive at-sea regulations while trawl benefits from the least restrictive, even without considering the implications for trawl's main target fisheries.

\hypertarget{robustness-operating-model-set}{%
\subsubsection{Robustness operating model set}\label{robustness-operating-model-set}}

Unlike the reference OMs, in which biomass and catch increases are large over the next decade, sablefish biomass and catch projections under the robustness OMs increased more gradually and generally required lower fishing rates. In fact, these simulations closely resemble previous sablefish MSE results in suggesting relatively conservative harvest strategies.

Tuning MPs to meet Objective 3 under the robustness OMs was more challenging because higher Fs had more noticeable impacts on the short term decline objective (P(decline); Table X). MP tuning produced relatively low target fishing mortality rates ranging from 5.24\% (current MP) to 6.7\% (full retention). Whereas the probability of catches less than 1,992 t (Objective 4) were negligible (\textless{} 3\%) in the reference OMs, they were all greater than 17\% in the robustness OMs except for full retention, which was 8\% (Table X).

Average annual catch under the robustness OMs ranged from 2,310 t under the current MP (MP15, noCap) to 2,770 t under full retention. Thus, the current MP with no limit on at-sea releases performed worse than any of the cap options, although most of the differences between cap options and the current MP were 80 t or less (Table X). There was a slight difference in the rank order of MPs (ranked by average 10-year catch) under the robustness OMs, although the absolute difference among most MPs was small.

Annual TACs were also more sensitive to variation in the MP data as indicated by higher average annual variation in catch (AAV), which was 14-16\% under the robustness OMs compared to 7-9\% under the reference OMs. This probably occurs because the stock is sometimes assessed below BMSY, leading to changes in both stock status and the maximum target fishing mortality, which has been relatively common in realized applications of sablefish MPs.

Cumulative 10-year landed value under the robustness OMs was approximately 60\% of landed value in the reference OMs. Although the absolute scales differ, the cumulative value patterns were similar to the reference set; that is, full retention produced the highest overall value, as well as value in each fleet, and the next best at-sea release regulation option was the least restrictive for trawl and next-most-restrictive for trap and longline hook.

\hypertarget{conclusions}{%
\section{Conclusions}\label{conclusions}}

This paper provides an evaluation of candidate management procedures for the Canadian sablefish fishery given updates to the operating models and possible addition of at-sea release regulations.

Stock status: depends on recruitment. Other fisheries deal w similar challenges (e.g.~hake, GOA sablefish)

MP evaluation: as indicated in previous MSE work, full retention of sablefish results in the highest average annual catch, while still allowing the fishery to meet conservation objectives in both the short and long term. Landed value is also greatest for a full retention option.

MPs 14,3,7 ranked among the top-3 under both reference and robustness OMs. However, tuning results in somewhat difference target Fs under MP7 F\_ref=0.068, F\_rob=0.059. This discrepancy could be handled by averaging the reference and robustness OMs to account for the high uncertainty in Year 2016 recruitment. Revisions to the strategy could be made in the next MSE cycle when that recruitment should be better estimated.

In the absence of changes to at-sea regulations, the current MP could probably stay the same, since F\_ref=0.0589 vs F\_rob=0.0524. The average would be somewhere near the current target of 5.5\%.

\hypertarget{contributors}{%
\section{Contributors}\label{contributors}}
\begin{longtable}[]{@{}ll@{}}
\toprule
Name & Affiliation\tabularnewline
\midrule
\endhead
Sean Cox & Simon Fraser University, BC\tabularnewline
Samuel Johnson & Simon Fraser University, BC\tabularnewline
Brendan Connors & DFO Science, Pacific Region\tabularnewline
\bottomrule
\end{longtable}
\MakeApproval

\hypertarget{sources-of-information}{%
\section{Sources of information}\label{sources-of-information}}

\hypertarget{refs}{}
\leavevmode\hypertarget{ref-cox2019evaluating}{}%
Cox, S., Holt, K., and Johnson, S. 2019. Evaluating the robustness of management procedures for the Sablefish (\emph{Anoplopoma fimbria}) fishery in British Columbia, Canada for 2017-18. Can. Sci. Adv. Sec. Res. Doc (032): vi + 79 p.

\leavevmode\hypertarget{ref-cox2009evaluation}{}%
Cox, S., and Kronlund, A. 2009. Evaluation of interim harvest strategies for sablefish (anoplopoma fimbria) in british columbia, canada for 2008/09. DFO Can. Sci. Advis. Sec. Res. Doc 42.

\leavevmode\hypertarget{ref-cox2011management}{}%
Cox, S., Kronlund, A., and Lacko, L. 2011. Management procedures for the multi-gear sablefish (anoplopoma fimbria) fishery in british columbia, canada. Can. Sci. Advis. Secret. Res. Doc 62.

\leavevmode\hypertarget{ref-cox2008practical}{}%
Cox, S.P., and Kronlund, A.R. 2008. Practical stakeholder-driven harvest policies for groundfish fisheries in british columbia, canada. Fisheries Research 94(3): 224--237. Elsevier.

\leavevmode\hypertarget{ref-cox2013roles}{}%
Cox, S.P., Kronlund, A.R., and Benson, A.J. 2013. The roles of biological reference points and operational control points in management procedures for the sablefish (anoplopoma fimbria) fishery in british columbia, canada. Environmental Conservation 40(4): 318--328. Cambridge University Press.

\leavevmode\hypertarget{ref-dfo2014performanc}{}%
DFO. 2014. Performance of a revised management procedure for sablefish in british columbia. Can. Sci. Adv. Sec. Res. Doc (025).

\leavevmode\hypertarget{ref-DFO2009}{}%
Fisheries and Oceans Canada. 2009. Summary of historic catch vs available weight. Pacific region fisheries management: Groundfish.

\MakeAvailable

\end{document}
